\documentclass[rmp]{revtex4}
\usepackage[english]{babel}
\usepackage{amssymb,amsfonts,amsmath}
\usepackage{xcolor}
\usepackage{breqn}
\usepackage{graphicx}
\usepackage{caption}
\usepackage{subcaption}
\captionsetup{compatibility=false}

\usepackage{amsmath}
\usepackage{natbib}
\usepackage{hyperref}
\usepackage{enumerate}
%\usepackage{subfig}

\usepackage{fontspec}

% Code syntax highlighting
\usepackage{listings}
\lstloadlanguages{C++,Python}

\newcommand{\EQ}[1]{Eq.~\eqref{eq:#1}}
\newcommand{\EQS}[2]{Eqs.~\eqref{eq:#1} and \eqref{eq:#2}}
\newcommand{\FIG}[1]{Fig.~\ref{fig:#1}}
\newcommand{\TAB}[1]{Tab.~\ref{tab:#1}}
\newcommand{\REF}[1]{ref.~\citep{#1}}

\newcommand{\comment}[2][noinline]{\todo[color=red,#1]{#2}}

\definecolor{dkgreen}{rgb}{0,0.6,0}
\definecolor{gray}{rgb}{0.5,0.5,0.5}
\definecolor{mauve}{rgb}{0.58,0,0.42}

%%% math macros
\DeclareMathOperator*{\argmin}{arg\,min}
\newcommand{\pfix}{p_{\mathrm{fix}}}

\usepackage{setspace}
\usepackage[colorinlistoftodos]{todonotes}

%%%%%%%%%%%%%%%%%%%%%%%%%%%%%%%%%%%%%%%%%%%%%%%%%%%%%%%%%%%%%%%%%%%%%%%%%%%%%%
\begin{document}
%%%%%%%%%%%%%%%%%%%%%%%%%%%%%%%%%%%%%%%%%%%%%%%%%%%%%%%%%%%%%%%%%%%%%%%%%%%%%%
\title{Genetic draft and valley crossing}
\author{Taylor~Kessinger}
\author{Jeremy~Van~Cleve}
%\affiliation{Department of Biology, University of Kentucky}


\date{\today}

%%%%%%%%%%%%%%%%%%%%%%%%%%%%%%%%%%%%%%%%%%%%%%%%%%%%%%%%%%%%%%%%%%%%%%%%%%%%%%
\begin{abstract}

Living systems are characterized by complex adaptations, at least some of which have arisen by evolutionary paths whose intermediate states are neutral or even deleterious.
Such adaptations have been termed ``irreducibly complex", and the process by which they evolve is known as ``crossing a fitness valley".
Previous efforts have rigorously characterized the rate at which such complex adaptations evolve in populations of roughly equally fit individuals.
However, populations that are very large or have broad fitness distributions, such as many microbial populations, adapt quickly, which substantially alters the fate of individual mutations.
We investigate the rate at which irreducible complexity evolves in these rapidly adapting populations, focusing on asexual populations.
We confirm that rapid adaptation overall increases the time required to cross a valley but can make it easier for deeper valleys to be crossed.
\end{abstract}

\maketitle

\begin{spacing}{1.5}
  
\section*{Introduction}

The simplest adaptive scenarios in evolution involve the arisal and fixation of successive beneficial mutations.
This appears to be how Darwin thought even complex systems, such as the vertebrate eye, evolved \citep{darwin_1859}, and this assumption underlies many models in evolutionary theory including the those from adaptive dynamics \citep{Geritz:Kisdi:1998,Dercole:Rinaldi:2008} and theories of adaptation \citep{Gillespie:1983,Orr:1998,Gillespie:1991}.
Evolution will certainly proceed in this fashion if there always exists a sequence of mutations from an initial genotype to the high fitness genotype where each mutant genotype in the sequence has a higher fitness than the previous one.
If the individual fitness of each genotype is visualized as a surface or landscape where the axes represent alternative alleles at each locus, then smooth landscapes with a single peak ensure that uphill paths exist no matter where a population starts. However, many empirical fitness landscapes are not completely smooth and have multiple peaks \citep[reviewed in][]{Szendro:Schenk:2013,Visser:Krug:2014,Obolski:Ram:2018}.
In such landscapes, wild type individuals may have to traverse a mutational valley--a region of lower fitness--in order to reach a higher fitness peak.
We seek to characterize this ``valley crossing" process in order to better understand the routes adaptation is likely to take on these rugged empirical landscapes \citep[e.g.,][]{Aguilar-Rodriguez:Payne:2017} and to assess the likelihood of valley crossing under more complex scenarios in which it may in fact be common \citep[e.g.][]{trotter_2014}.

Valley crossing in asexual populations is increasingly well understood \citep{weissman_2009}.
If the number of mutants produced every generation is high, which occurs in large populations with high mutation rates, a complex adaptation requiring multiple mutations that is destined to fix will eventually be generated de novo. This is sometimes called deterministic fixation of the double mutant \citep{weissman_2009}.
In small populations, neutral or deleterious intermediate mutants generated by wild type individuals can drift to fixation. Then, the further beneficial mutations necessary for the adaptation can fix due to positive selection. This is the sequential fixation regime.
A third possibility occurs for intermediate population sizes where wild type individuals generate transient mutant subpopulations or ``bubbles".
These bubbles are ordinarily doomed to extinction due to drift, but additional mutations in a lucky bubble may generate a complex adaptation that sweeps to fixation; this process has been referred to as ``tunneling" \citep{iwasa_2004, weissman_2009}.
On its own, valley crossing becomes more likely as the population size increases \citep{weissman_2009}. However, recent work suggests that valley crossing relative to the fixation of a simple beneficial mutation is least likely at intermediate population sizes where tunneling occurs \citet{ochs_2015}. This suggests that tunneling may be the most difficult mode of valley crossing. 
% Valley crossing in sexually reproducing populations was by \citet{weissman_2010}, which found that the criterion $r < s$, with $r$ the recombination rate and $s$ the selective advantage of the complex adaptation, is the necessary criterion for recombination to be helpful in crossing a valley.
% Fairly low recombination rates can bring mutant subpopulations together, increasing the rate at which complex adaptations arise and sweep.
% High recombination rates, on the other hand, lower the valley crossing rate by forcing fit individuals to outcross with the less fit wild type.

Previous studies of valley crossing focus primarily on populations where all individuals are equal in fitness except for the focal loci, i.e., the loci at which the individual mutations comprising the complex adaptation segregate.
This is tantamount to assuming that, if there is variation in the background fitness of the population, it is negligible.
In such populations, genetic drift governs the fate of neutral alleles, as well as the behavior of deleterious or beneficial alleles where they are rare, close to frequency zero, or common, close to frequency one.
Alleles whose dynamics are primarily determined by genetic drift can be effectively modeled by a diffusion approximation \citep{Wright:1945,Kimura:1955,Kimura:1957}, and the ancestry of a population is described well by the classic Kingman coalescent \citep{Kingman:1982} where pairs of branches in the genealogy merge back in time until the most recent common ancestor.
When population sizes are very large or the fitness variation in the population is substantial, however, the behavior of genetic variation is governed more by genetic \emph{draft} than by genetic drift \citep{gillespie_2000, gillespie_2001, masel_2011, neher_shraiman_2011}.
Whereas drift is the effect of imperfect sampling from generation to generation, draft is the sum of hitchhiking effects due to selection on sites linked to the focal loci.
Draft is fundamentally different from drift: for example, drafting populations experience large jumps in allele frequencies that cannot be encapsulated by a diffusion approximation \citep{neher_shraiman_2011}
and have genealogies where more than two branches can merge at once and that are better described by alternative coalescent processes such as the Bolthausen-Sznitman coalescent \citep{neher_hallatschek_2013, brunet_2007, schweinsberg_2017}.
Additionally under genetic draft, the frequency spectrum of neutral alleles is non-monotonic, with a marked uptick near frequency one--or, equivalently, a depression in intermediate frequency alleles, which are quickly swept out of the population \citep{neher_shraiman_2011,kosheleva_2013,neher_hallatschek_2013}.
Only a handful of individuals near the ``nose" of the fitness distribution are likely to persist, and they give rise to the bulk of the future population and carry linked alleles with them.
The parameter that determines whether drift or draft is more important is the product of the population size $N$ and the standard deviation in fitness $\sigma$ \citep{neher_hallatschek_2013}.

It is increasingly realized that draft may play a critical role in shaping genetic diversity, especially in many microbial species, where population sizes can be large but ``effective population sizes" are many orders of magnitude smaller \citep{masel_2011}.
The possibility that neutral variants are affected more strongly by selection at linked sites than by genetic drift, even in organisms like \emph{Drosophila}, is one possible explanation for the ``paradox of variation", the fact that genetic diversity and population size often do not scale linearly \citep{gillespie_2000, gillespie_2001, neher_kessinger_2013, corbett-detig_2015}.
Draft likewise may have profound effects on the evolution of complex adaptations. Preliminary evidence for this comes from
\citet{neher_shraiman_2011} who explored how draft affects valley crossing via stochastic tunneling by calculating the mutant bubble size distribution in a rapidly adapting population.
They found that, compared to drift, draft generally makes small and intermediate sized mutant bubbles much rarer but very large bubbles are much more common. Thus, while draft may make valley crossing via tunneling more difficult, the total effect isn't immediately clear.
Here, we extend these previous results by studying in asexual populations how genetic draft affects the rate of crossing fitness valleys across a range of population sizes.

The complicating factor is that, in an asexual population, the dynamics are governed almost entirely by what happens in the nose of the fitness distribution, and these dynamics can be highly stochastic. In previous approaches, these stochastic effects were smoothed due to the presence of recombination \citep{neher_shraiman_2011}.
We therefore must focus primarily on simulation approaches as analytical solutions for the behavior of a mutation in the nose of an asexual population are difficult to obtain.
We show that fitness valley crossing occurs at an overall lower rate in rapidly adapting populations.
However, this is consistent with the fact that \emph{all} forms of adaptation are slowed down.
In addition, we confirm that in rapidly adapting populations, adaptive fixation of alleles are more likely to be complex adaptations involving a fitness valley than they are in populations where genetic drift is the primary force shaping genetic variation at linked sites; essentially, genetic draft maintains the linkage that makes it possible to leap across fitness landscapes rather than adapt primarily by climbing to local peaks.
These observations add to the growing intuition that complex adaptations that involve evolution across fitness valleys may not be a surprising result of the evolutionary process, but rather an expected one.

\section*{Mathematical background}

\begin{figure}
\begin{center}
  \includegraphics[scale=0.75]{Figures/valley.pdf}
  \caption{Fitness landscape.}
  \label{fig:landscape}
\end{center}
\end{figure}

Our model features a population of $N$ haploid individuals in which an effectively infinite number of ``background" loci, with weak fitness effects, are currently segregating.
We focus on two loci, which are initially fixed for alleles $a$ and $b$.
These alleles mutate to $A$ and $B$ at rates $\mu_A$ and $\mu_B$, respectively (we neglect back mutations).
Genotypes $ab$, $Ab$, $aB$, and $AB$ have fitnesses $1$, $1-\delta_A$, $1-\delta_B$, and $1+s$ respectively, with $\delta \geq 0$ (intermediate genotypes are deleterious) and $1/N < s < 1$ (double mutant $AB$ is strongly beneficial).
For the remainder of our analysis, we will assume that $\mu_A = \mu_B = \mu$ and $\delta_A = \delta_B = \delta$.
We further assume that the large number of loci of weak effect contribute to a constant background fitness variance $\sigma^2$ which, by Fisher's ``fundamental theorem" of natural selection \citep{Fisher:1930}, sets the rate $v$ at which the mean fitness advances to $\sigma^{2}$.
If $x$ is the background fitness of a lineage, then the distribution of background fitnesses $f(x)$ is assumed to be roughly Gaussian: $f(x) = \frac{1}{\sqrt{2\pi}\sigma} \exp (\frac{(x-\bar{x})^2}{2\sigma^2})$.
Given this background fitness variance, we are interested in the expected time $\mathbb{E}\left[ T\right]$ for the double mutant $AB$ (our so-called ``complex adaptation'') to arise and fix in the population.

In very large populations, enough double mutants are generated that the valley crossing time is dominated by the time required for a double mutant to sweep through the population.
At the other extreme, in small populations, the crossing time is dominated by the wait for a deleterious single mutant to fix: only thereafter does a successful double mutant appear.
We begin by zeroing in on the intermediate case: stochastic tunneling, the most interesting and mathematically demanding mode of valley crossing.
We will review existing theory for this case, discuss the difficulty in the mathematical analysis due to genetic draft, and then present results for valley crossing times using forward-time simulations across a range of population sizes that include the sequential fixation, tunneling, and deterministic fixation cases.

There are three components to the process of stochastic tunneling.
First, a single mutant lineage must appear, which occurs with rate $N\mu$.
If this lineage arises at time $t_0$ and persists until time $T$, then it gives rise to $W = int_{t_0}^T n(t) dt$ total individuals before going extinct, where $n(t)$ is the number of mutant individuals extant at time $t$.
We refer to such a short-lived mutant lineage, and all the individuals generated within that lineage, as a mutant ``bubble": the time integrated number of individuals $W$ is the ``weight" of the bubble.
During the lifetime of a bubble, the complex adaptation must appear and establish; that is, it must rise to a high enough frequency that fixation is almost guaranteed, which occurs at rate $\mu \pfix$ where $\pfix$ is the probability that the double mutant reaches fixation from a single initial individual.
If it establishes, the complex adaptation will quickly sweep to fixation.
The first two steps can be folded in together and considered as one process, so that what is relevant is the expected time until a single mutant lineage arises that is destined to produce a successful double mutant.
In this way, $\mathbb{E}\left[ T \right] = \mathbb{E} \left[ T_0 + T_1 \right]$, where $T_0$ is the time to the first successful bubble--the first that produces a double mutant that is destined for fixation--and $T_1$ is the time required for the double mutant to sweep.
We assume that $T_0 \gg T_1$, so that the wait time is dominated by the first term
\comment{JVC: justify this briefly}.

Most of the time, a bubble will simply arise and go extinct.
If mutation and fixation are sufficiently rare events, then fixation of the double mutant can be modeled as a Poisson process: with probability $1-e^{-\mu \pfix W}$, the bubble with total size $W$ will give rise to a successful double mutant.
Computing the expected value of this quantity over all bubble sizes, $\phi = \left< 1-e^{\mu \pfix W} \right>$, gives the probability that some bubble will lead to a successful valley crossing.
One difficulty in computing the rate of tunneling in the high $N\sigma$ case where genetic draft is operating is that the expected bubble size and fixation probability will both turn out to depend on the background fitness of an individual.
That is to say, it is not possible to simply compute $\pfix$ and the weight distribution by themselves, but rather one must convolute both of them over the background fitness.
Individuals whose background fitness is near the nose of the fitness distribution will tend to give rise to longer-lived lineages, and if those individuals carry the complex adaptation, it will therefore likewise be more likely to fix.
Thus, $W$ will depend on the background fitness $x$ and the valley depth $\delta$, and $p_{\mathrm fix}$ will depend on $x$ and the peak height $s$.
These will have to be integrated over the fitness distribution $f(x)$ in order to obtain the total crossing probability $\Phi(s,\delta) = \int_{-\infty}^\infty \phi(x,s,\delta) f(x) dx$.
The expected valley crossing time then becomes $1/N\mu\Phi$.

\comment[inline]{
  \begin{itemize}      
  \item  This section is still too technical. For example, the reader here doesn't need to know what  ``weak driver'' is, or if they do, it should be more fully explained and used more than just in what is effectively a background section. 
  \item I think having the PDE for the fixation probability is ok, but you don't need the details about the limits necessary for deriving it. The point of keeping it more informational for the reader. 
  \item Also, avoid introducing new variables, like $D$, and keep the existing notation, like $\pfix$, instead of changing to $p$ since the simplicity doesn't save much time (we only use the equation once) and the extra notation could be confusing. 
  \item Part of the motivation for introducing the PDE in the first place is to then say what the results of analyzing it, namely what the fixation probability is under draft with fitness background $x$. You explain this a little bit, but you can more fully describe how the $\pfix$ increases exponentially then linearly. 
        \item Also, I don't think you mention how the valley depth $\delta$ and peak height $s$ related to the background fitness $x$ and fitness variance $\sigma$. This should probably come up here, but also probably in the first paragraph of the ``Mathematical background''
        \end{itemize}
      }
{\color{red}
We begin by considering the dynamics of $\pfix(x,s)$ before moving on to consider the distribution of bubble sizes.
We zero in on the ``weak driver" model, in which background mutations are frequent but of weak effect compared to $\sigma$.
In this case, the background beneficial mutation rate $U_b$ and mean fitness effect $\bar{\epsilon}$ turn out not to matter: what is relevant is the product $D = U_b \bar{\epsilon}^2/2$.
Taking the limit $U_b \to \infty$ and $\bar{\epsilon} \to 0$ yields the ``infinitesimal" model.
\citet{hallatschek_2011} and \citet{good_2014} showed that the fixation probability in this regime obeys the differential equation 
\begin{equation}
v\partial_x p(x,s) = x p(x,s) + D\partial_x^2 p(x,s) - \frac{p(x,s)^2}{2},
\end{equation}
in which we have assumed that $v \approx \sigma^2$, which is tantamount to claiming that mutational effects are less important than selection in determining the advance of the mean fitness.
The fixation probability then increases linearly in $x$ above some critical value $x_c$ and remains small otherwise, with $x_c \approx \sigma^4/4D$: when $\sigma$ is large, only mutations in the nose stand any chance of fixing unless they can be dragged to higher fitness by a background mutations (large $D$).
}

In order to compute the total valley crossing probability $\Phi$, the only remaining piece to compute is distribution of mutant bubble sizes $W$ as function of their background fitness $x$.
Since the tunneling probability for a lineage with background fitness $x$ is given by $\phi = \left< 1 - e^{-\mu p_{\mathrm fix} W} \right>$, we do not need to compute the full distribution of $W$ but rather only its Laplace transform: $\mathcal{L}\left[ p(w) \right] = \int_0^\infty e^{-zw} p(w) dw$, with $z = \mu \pfix$.
\citet{neher_shraiman_2011} showed that $\phi$ follows the equation
\begin{equation}
\partial_t \phi = z + (z + x - \sigma^2 t + \delta)\phi^2 - (1 + x - \sigma^2 t + \delta)\phi^2,
\end{equation}
but their analysis also included a recombination term that made it possible to pull out the integrated Laplace transform $\int_{-\infty}^\infty p(x) \phi(x) dx$ without the need to actually solve this equation or send time to infinity.
Unfortunately this method is not available to us, and the nose dominance of the bubble size distribution, combined with the $x$ dependence of $z$ (via $z = \mu \pfix(x)$), lead to difficulty in deriving a simple closed form expression for $\Phi$.
Therefor, we focus on simulation methods to study the dynamics of valley crossing.

\section*{Simulation methods}

To analyze our model, we perform forward simulations using a customized version of \texttt{FFPopSim} \citep{zanini_2012}, a discrete time forward simulation package that implements a modified Wright-Fisher model.
The simulation code is available via \texttt{GitHub}.
In \texttt{FFPopSim}, individuals are organized by ``clones" (sets of individuals with the same genome): each individual gives rise to a Poisson distributed number of offspring, with the mean dependent on the relative fitness.
The mean offspring number is further adjusted to keep the population roughly at a pre-specified carrying capacity.
We initialize a wild type population of $N$ individuals consisting of haploid genomes of length $L = 200$.
Genetic loci are partitioned into two groups, the two ``focal" loci (at which the epistatic alleles will segregate) and ``background" loci (which are responsible for the underlying fitness distribution).
For the background loci, we make use of a modified ``infinite sites" model: any time a locus becomes monomorphic, a mutation at that locus is injected into a random individual in the population.
In this way, the population experiences a constant influx of beneficial mutations at the background loci.
These mutations' fitness effects are drawn from an exponential distribution; the background fitness variance is set to a constant value $\sigma^2$ by manually adjusting the selection coefficients every generation (they are multipled by the current variance and divided by $\sigma^2$).
In this way, the background mutation rate $U$ and average fitness effect $\bar{\epsilon}$ are not parameters of our simulation model but must be directly measured from simulations: they are constrained by $L$ and $\sigma$.
We estimate them by manually counting the number of injected mutations every generation and averaging over the fitness effect at each locus.

We set the ``focal" loci (the loci at which the epistatic alleles segregate) to be at positions $L/4$ and $3L/4$.
Simulations are allowed to proceed for an ``equilibration" time of $N/10$ generations to allow sufficient genetic diversity to be introduced.
During this time, the mutation rate $\mu$ for the focal loci is set to zero.
After equilibration, we set a per-site mutation rate $\mu$ for the focal loci (which controls both forward and backward mutations), with single mutant fitness disadvantage $\delta$ and double mutant advantage $s$, consistent with our analytical model.
When the double mutant reaches frequency $0.5$, we consider it to have fixed.
We then set the focal loci allele frequencies back to zero, randomize the selection coefficients for the genetic background by drawing anew from an exponential distribution, and allow the population to equilibrate for another $N/10$ generations.
This offers some assurance that independent valley crossing trials are indeed independent despite occurring in the same simulation run.

We partitioned our simulations as follows.
We first studied the effect of increasing $\sigma$ in what \citet{weissman_2009} describes as the sequential fixation, stochastic tunneling, semi-deterministic tunneling, and sequential fixation regimes.
In the case of sequential fixation, the valley crossing time is dominated by the time until a deleterious mutant that is destined to fix arises: once it does, it does so in order $T_{\mathrm{MRCA}}$ generations, which is of order $N$ for low $N\sigma$ and scales weakly with $N$ for higher $N\sigma$ \citep{neher_hallatschek_2013}.
In the drift limit, sequential fixation applies when $N \ll \mathrm{min}(1/\delta, 1/\sqrt{\mu s})$, so that a successful mutant is unlikely to arise until the intermediate has fixed, but the intermediate's fitness disadvantage is too weak for selection to inhibit it.
Next, we considered the stochastic tunneling regime.
Tunneling becomes relevant when $N$ is too large for sequential fixation, i.e., the tunneling probability becomes more significant than the probability that the intermediate will fix: this mandates $N \gtrsim 1/\sqrt{\mu s}$.
It also requires that $\ll 1/\mu$, meaning that only one mutant lineage is likely to be extant at any given time and, if they do co-occur, are unlikely to interfere with each other or substantially affect the mean fitness.
This happens to be the necessary condition for a birth-death process to be an appropriate model.
By increasing $N$ far above the $1/\mu$ limit, multiple deleterious mutants are likely to occur every generation, reaching mutation-selection balance at low frequency and providing many opportunities for a lucky double mutant to appear (so-called ``semi-deterministic tunneling").
At still higher population sizes ($\gtrsim s/\mu^2$), multiple double mutants are likely to appear every generation, and their dynamics can be modeled as deterministic.

Finally, we compared the dynamics of valley crossing versus sweeping mutations.
We performed simulations with a sweeping focal beneficial mutation, with selection coefficient $s_{\mathrm{sweep}}$, at position $L/2$, with the background loci kept polymorphic as in the valley crossing simulations.
We recorded the time for a beneficial mutation to arise and reach frequency $0.5$, considering it to have fixed at that point.
We then computed the ratio of the sweep time to the valley crossing time for a comparable complex adaptation, with valley depth $\delta$ and larger fitness advantage $s_{\mathrm{valley}}$.
To determine the effect of $\sigma$ on this ratio, we performed simulations at both high and low $\sigma$ and computed the ratio of these ratios: the result is a measure of the extent to which rapid adaptation favors valley crossing over simple sweeps.

\section*{Results}

\begin{figure}
\begin{subfigure}[b]{0.4\textwidth}
\includegraphics[width=\textwidth]{Figures/det_tunnel.pdf}
\end{subfigure}
\begin{subfigure}[b]{0.4\textwidth}
\includegraphics[width=\textwidth]{Figures/det_fix.pdf}
\end{subfigure}
\caption{Valley crossing times in the semi-deterministic tunneling (left) and deterministic fixation (right) regimes. In both figures, $s = 10^{-2}$ and $\delta = 10^{-3}$: on the left, $N = 10^5$ and $\mu = 10^{-4}$, and on the right, $N = 10^6$ and $\mu = 10^{-3}$. Orange lines are averages, boxes are interquartile ranges, and end caps are overall ranges. Each box represents about $100$ simulation runs.}
\label{fig:deterministic}
\end{figure}


\begin{figure}
\begin{subfigure}[b]{0.4\textwidth}
\includegraphics[width=\textwidth]{Figures/seq_fix.pdf}
\end{subfigure}
\begin{subfigure}[b]{0.4\textwidth}
\includegraphics[width=\textwidth]{Figures/neut_tunnel.pdf}
\end{subfigure}
\caption{Valley crossing times in the sequential fixation (left) and tunneling (right) regimes. In both figures, $s = 10^{-2}$ and $\delta = 10^{-3}$: on the left, $N = 10^4$ and $\mu = 3\times 10^{-5}$, and on the right, $N = 10^5$ and $\mu = 3\times 10^{-6}$. Labeling is as in figure \ref{fig:deterministic}.}
\label{fig:tunneling}
\end{figure}

Here we present the key results of our simulation runs.
In general, we expect valley crossing in rapidly adapting populations to require more time, as bubbles are doomed to extinction more quickly and beneficial lineages are less likely to fix, especially when valleys are shallow.
Figures \ref{fig:deterministic} and \ref{fig:tunneling} confirm these expectations: the overall rate of valley crossing is generally slowed as $\sigma$ increases.
It is tempting to interpret this as a reduced $N_e$ due to a decrease in genetic diversity and the strength of selection.
The underlying dynamics are more complicated, however.
As $\sigma$ increases, the effect of the genetic background becomes more significant in determining the fate of an allele than the allele's fitness effect is.
The frequency of alleles is buffeted by genetic draft, so alleles behave as though they are more neutral than they really are.
This depresses the fixation probability of the complex adaptation, which appears to account for the majority of the change in fixation time.
In the case of stochastic tunneling, the overall size distribution of bubble sizes is reduced, as well: the distribution $P(w)$ scales not as $w^{-3/2}$ but as $w^{-2}$ \citep{neher_shraiman_2011}, so there are fewer chances for the complex adaptation to arise and fix.

\begin{figure}
\includegraphics[width=0.4\textwidth]{Figures/var_sigma_delta.pdf}
\caption{Valley crossing times for varied $\sigma$ and $\delta$. In these simulations, $N = 10^5$, $\mu = 3 \times 10^{-6}$, and $s = 10^{-2}$. Note the lack of dependence on $\delta$ at high $\sigma$.}
\label{fig:sigma_delta}
\end{figure}

On the other hand, the reduced effectiveness of selection should generally be helpful in crossing deeper valleys, by mitigating the deleterious effect of single mutants.
This can be seen in figure \ref{fig:sigma_delta}.
At high values of $\sigma$, the fitness effect of the deleterious intermediate \emph{almost does not matter}, meaning that very deep valleys can be crossed.
This supports the view that traversing rugged fitness landscapes by valley crossing may be enhanced in rapidly adapting populations.

% \iffalse
% \begin{figure}
% \includegraphics[width=0.4\textwidth]{Figures/bands.pdf}
% \caption{Valley crossing times for fixed $N\sigma$, with $N$ allowed to vary up or down by a factor of $2$ (and $\sigma$ adjusted accordingly). Error bars are quartile ranges. Here, $\mu = 3 \times 10^{-6}$, $\delta = 10^{-3}$, and $s = 10^{-2}$.}
% \label{fig:bands}
% \end{figure}

% In addition, we consider the effects of the compound parameter $N\sigma$, which has previously been shown to determine the coalescent behavior of populations \citep{neher_hallatschek_2013, neher_kessinger_2013}.
% Figure \ref{fig:bands} suggests that, at high values of $\sigma$, the value of $N\sigma$ is more important than $N$ itself in determining the valley crossing rate.
% The dependence on $N$ appears to be reduced, consistent with general features of very large, rapidly adapting populations.
% In a slowly adapting population, $N$ determines the transition from sequential fixation to tunneling: if bubbles reach a maximum size that is of order $N$, they start to noticeably decrease the mean fitness, making fixation easier.
% When $N\sigma \gg 1$, bubbles almost never reach a large enough size for this condition to obtain.
% \fi

\begin{figure}
\begin{subfigure}[b]{0.43\textwidth}
\includegraphics[width=\textwidth]{Figures/tau_compare.pdf}
\end{subfigure}
\begin{subfigure}[b]{0.4\textwidth}
\includegraphics[width=\textwidth]{Figures/alpha_ratios.pdf}
\end{subfigure}
\caption{Behavior of valley crossing times for varied $\delta$, $N$, and $\sigma$. In all cases, $s_{\mathrm{sweep}} = 0.01$, $s_{\mathrm{sweep}} = 0.1$, and $\mu = 10^{-5}$. Left two plots: valley crossing times $\tau$ increase with $\delta$ and $\sigma$ but decrease with $N$, as predicted. Right two plots: the ratio $\alpha = \tau_{\mathrm{valley}}/\tau_{\mathrm{sweep}}$ as a function of $\delta$, $N$, and $\sigma$. Deeper valleys in smaller populations are disfavored relative to simple sweeps. But at higher values of $\sigma$, they can be favored ($\alpha < 1$: note the log scaled colorbar). In both plots, $N = 10^5$ is a fairly hard cutoff beyond which tunneling becomes semi-deterministic, as $N\mu > 1$ means that, on average, more than one mutant lineage is seeded every generation.}
\label{fig:ratios}
\end{figure}

\begin{figure}
\includegraphics[width=0.4\textwidth]{Figures/compare_ratios.pdf}
\caption{Ratio of $\alpha$ values at high and low $\sigma$. Genetic draft is helpful for crossing deep valleys in the sequential fixation and tunneling regimes. But when semi-deterministic tunneling becomes more important than stochastic tunneling, it can slow down the crossing rate.}
\label{fig:alpha_ratio}
\end{figure}

Finally, we examine the ratio $\alpha = \tau_{\mathrm{valley}}/\tau_{\mathrm{sweep}}$, where $\tau_{\mathrm{sweep}}$ is the time required for a beneficial mutation to sweep to fixation and $\tau_{\mathrm{valley}}$ is the time required for a valley crossing event to occur.
By comparing how $\alpha$ changes as a function of the strength of selection on linked variation, $\sigma$, we can determine how $\sigma$ affects the speed of valley crossing relative to simple beneficial sweeps. Higher (or lower) values of $\alpha$ mean that valley crossing times increase (or decrease) relative to the time required for a simple beneficial sweep.
Though the precise value of this ratio depends on $\delta$ and the selection coefficients $s_{\mathrm{valley}}$ of the complex adaptation and $s_{\mathrm{sweep}}$ of the sweeping solo beneficial mutation, a general trend nonetheless emerges.
In this case, we set $s_{\mathrm{sweep}} = 0.01$, $s_{\mathrm{sweep}} = 0.1$, and $\mu = 10^{-5}$.
For several combinations of parameter values, we find that the ratio $\alpha$ is smaller for high $\sigma$ than for low $\sigma$: see figure \ref{fig:ratios}.
In figure \ref{fig:alpha_ratio}, we consider the behavior of $\alpha(10^{-2})/\alpha(10^{-6})$, a measure of the extent to which rapid adaptation speeds up valley crossing relative to sweeping (lower values indicate that rapid adaptation favors valley crossing).
Overall, we find that increasing $\sigma$ increases the extent to which valley crossing is favored \emph{except} when $N\mu > 1$.
In this case, stochastic tunneling ceases to be the dominant mode of valley crossing.
\citet{weissman_2009} referred to this regime as ``semi-deterministic tunneling": a more or less stable number of intermediate mutant individuals $N\mu/\delta$ persists in the population at low frequencies, which give rise to double mutants at a low but constant rate.
At high $\sigma$, the number of intermediate mutants is likely to fluctuate substantially \citep{cvijovic_2017}, but the average behavior is more or less the same.
This relation more or less obtains in a rapidly adapting population as well, with substantial fluctuations, so $\pfix$ becomes more important in determining the valley crossing rate: since this scales more weakly with $s$ under draft, the overall crossing rate is slowed.

\section*{Discussion}

We have used simulation methods to examine the rate at which rapidly adapting populations cross fitness valleys.
Though the overall time required is generally higher than in slowly adapting (drifting) populations, there are two important caveats.
First, the effect of the deleterious intermediate is much smaller: for $\delta \lesssim \sigma$, the precise value of $\delta$ almost does not affect the crossing rate.
Second, the time to the evolution of even \emph{simple} adaptations (i.e., ones requring substitution only at a single locus) is likely to be much longer in rapidly adapting populations.
This means that the presence of a fitness valley is not as steep of an impediment to the evolution of a complex adaptation for a drafting population as it is for a drifting one.
This confirms the intuition that traversing a rugged fitness landscape by valley crossing may be favored in rapidly adapting populations.
These results compare favorably with those of \citet{ochs_2015}, which studied a similar problem in a slowly evolving population.
They considered the scenario where a sweeping beneficial mutation \emph{competes} with a particular complex (valley crossing) adaptation, whereas we treat these two scenarios separately, but the trend is similar: in their model, the complicating factor is that a partially sweeping beneficial mutation increases the mean fitness, making valley crossing more difficult especially for populations of intermediate size.

We have focused exclusively on asexual populations, but extensions to sexual populations are possible.
The relevant parameter is likely to be the product of $N$ and the proportion of the fitness standard deviation segregating in a small, effectively asexual block, $\sigma_b$ \citep{neher_kessinger_2013}.
If both $N\sigma$ and $N\sigma_b$ are large, and the focal loci segregate in the same block, then the analysis presented here should determine the crossing time.
If, on the other hand, $N\sigma$ is large but $N\sigma_b$ is small, or the distance between the focal loci is much larger than the block length $\xi_b$, then recombination between the focal loci can be modeled as ``free", with individuals effectively shuffling their entire genomes every generation.
In that case, the communal recombination analysis of \citet{neher_shraiman_2011} and \citet{neher_shraiman_2010} is likely to be important: the relationship between $\sigma$ and the recombination rate $\rho$ determines the crossing rate.
If $\sigma$ is large, then draft dominates the crossing rate: if $\rho$ is large, then drift dominates it.

%%% from into but more relevant here
% In sexual populations, the accuracy of this description depends on the recombination model.
% When recombination is ``free", meaning that child genomes are assembled by randomly shuffling those of their parents, there is a chance for new alleles on weak fitness backgrounds to jump onto fitter ones near the nose of the distribution \citep{neher_shraiman_2011}.
% Competition between recombination, which breaks up linkage, and adaptation, which creates it, determines whether draft or drift is more important.
% When recombination occurs via crossovers, genomes can be partitioned into effectively asexual ``blocks", whose size depends on the per-site crossover rate and the total fitness variation segregating in the population.
% If each block harbors some portion $\sigma_b$ of the total fitness standard deviation $\sigma$, then the conditions $N\sigma_b \gg 1$ and $N\sigma_b \ll 1$ correspond to draft and drift dominance, respectively \citep{neher_kessinger_2013}.


The dynamics by which rugged fitness landscapes are traversed has received significant attention in the literature in recent years.
Valley crossing has previously been shown to be favored in small populations, where the effects of selection against intermediates are mitigated, and in large populations, where multiple mutants are much more likely to appear \citep{weissman_2009, ochs_2015}.
Low amounts of recombination also appear to be helpful \citep{weissman_2010}: at high values of the recombination rate, double mutant individuals outcross with the wild type, producing more deleterious intermediates and retarding the growth of the double mutant.
An additional effect that can heighten the rate at which valley crossing occurs is population subdivision \citep{Bitbol:Schwab:2014}: the population size within a subdivision is smaller, meaning that selection against potentially harmful intermediates is relaxed.
Like population subdivision, genetic draft favors the appearance of complex adaptations by forcing selection to act on entire genotypes, which may be carrying useful but provisionally harmful deleterious intermediate mutations.
This can be contrasted with effects such as high migration rates and high recombination rates \citep{neher_shraiman_2009}, which force selection to act on individual alleles rather than genotypes.

Two further intuitions can be drawn from this argument.
First, there is a sort of duality between the evolution of complex adaptations and the evolution of cooperation, which can be seen as a complex behavioral adaptation.
In social evolution theory, high migration rates between demes, which cause organisms to be less likely to interact with close kin, disfavor the evolution of cooperation: likewise, when migration rates are low, cooperation can be favored \citep{van_cleve_2015}.
Evolutionary forces which strengthen existing associations between loci are more likely to lead to such complex phenotypes: evolutionary forces which weaken these associations hinder their evolution.
This reasoning applies whether the loci in question appear in the same individual (as in the case of sign epistasis) or in different individuals (as in the case of social evolution).
These effects can even interact: cooperation can function as an additional evolutionary force that favors valley crossing \citep{Obolski:Lewin-Epstein:2017} and can in some cases be a valley crossing adaptation in itself \citep{van_cleve_2013}.
The interplay between cooperation and valley crossing is an area that needs further study, as it may shed light on the evolution of cooperation within large microbial populations such as yeast \citep{gore_2009, gore_2013} and bacterial biofilms \citep{rainey_2003, van_gestel_2014}.

Second, the relationship between genetic draft and valley crossing has implications for Wright's ``shifting balance theory".
Population subdivision followed by colonization of a new fitness peak may be an important aspect of the evolutionary process, but it is not obvious \emph{how} important \citep{coyne_barton_turelli_2000}.
However, populations which experience both subdivision \emph{and} rapid adaptation are ideal candidates in which the shifting balance theory may apply.
Such populations include HIV, in which host-specific adaptations evolve quickly \citep{zhang_1997, wain_2007, dapp_2017, theys_2018} and in which deleterious mutations are known to hitchhike to high frequency \citep{zanini_2013, zanini_2015}.
Our work also shines light on the likely pathway through which multi-drug resistance evolves in HIV.
In recent years of the HIV pandemic, resistance has generally been slower to evolve \citep{feder_2015}, in general because more and stronger drugs are used.
There are two possible explanations for this.
One is that the use of anti-HIV drugs decreases the number of virions segregating within an individual, thus lowering the number of possible chances for a resistant phenotype to appear.
The other is that such drugs, especially when used in concert, contort the fitness landscape so that evolution of a phenotype that is resistant to a drug cocktail is more difficult.
Both factors are likely to play a role.
However, since HIV is a population in which draft, not drift, is the dominant factor and hence population size is less important in determining the evolutionary dynamics, our work suggests that the second is likely to be more significant compared to other populations.



%In the case where $\rho = 0$ and the fitness variance $\sigma^2$ is small ($N\sigma \ll 1$), the rate of valley crossing agrees with \citet{weissman_2009}: when $\rho > 0$, the rate agrees with \citet{weissman_2010}.
%This is because, provided that $N\sigma \ll 1$, genetic drift is the major evolutionary process affecting the frequency of deleterious intermediates, and the fixation probability of the complex adaptation simply scales with $s$.
%We turn to the regime where $N\sigma \gg 1$, meaning that draft is more important than drift, but the fixation probability of the full complex adaptation is suppressed.
%By the central limit theorem, the background loci create a background fitness wave whose shape is roughly Gaussian.
%We assume that the constant influx of weak effect beneficial mutations keeps the fitness variance $\sigma^2$ roughly constant.
%We sketch some heuristics in this limit before providing a more formal analysis.

% \iffalse
% First, we consider the case where $\rho = 0$ and focus on what \citet{weissman_2009} refer to as the process of ``stochastic tunneling", where $N\mu \ll 1$.
% As we shall see, the results in the $\rho = 0$ and low $\rho$ limits can easily be extended to sexually recombining populations: moreover, understanding the rate of tunneling will allow us to easily examine the rate at which sequential fixation and deterministic fixation occur.
% In stochastic tunneling, a deleterious intermediate arises (at rate $N\mu$), and its lineage persists over a short time $T$, forming a transient ``bubble".
% If the number of individuals in the lineage is given by $n(t)$, then the total number of individuals in the ``bubble" is $w = \int_{t_0}^T n(t) dt$.
% We refer to $w$ as the ``weight" of the bubble.
% The overall rate of valley crossing becomes $1/N\mu p$, with $p$ the probability that a particular bubble will give rise to a successful double mutant.
% Thus, given that a deleterious intermediate appears, it has $w$ ``chances" to generate a successful complex adaptation, which it does at rate $\sim \mu s$: double mutants arise at rate $\mu$ and have a probability $\sim s$ of fixing.
% The probability of achieving a weight of at least $w$ scales as $1/\sqrt{w}$ (or, equivalently, $P(w) \sim w^{-3/2}$), meaning that with probability $\sqrt{\mu s}$, a bubble gives rise to a successful single mutant.
% This probability drops off rapidly at $1/\delta^2$, so for higher values of $\delta$ (so called ``deleterious tunneling"), the rate at which new complex adaptations are generated starts to fall: the size of the mutant subpopulation is bounded from above by roughly $1/\delta$.

% In the high $N\sigma$ regime, the argument is much the same, but the underlying dynamics are different.
% In stark contrast to the drift case, the fate of a complex adaptation depends strongly on the fitness $x$ of the genetic background on which it appears.
% Individuals with fitness lower than some critical fitness $x_c$ are almost guaranteed to go extinct.
% This means the fixation probability of the complex adaptation no longer scales with $s$, and $P(w) \sim w^{-2}$ instead of $w^{-3/2}$, meaning that the probability of reaching weight $w$ is $\sim 1/w$.
% However, individuals with higher fitnesses can be destined for fixation, even if they are carrying a significant load of deleterious mutations.
% This means the dependence on $\delta$ is significantly lessened: even deep fitness valleys can be crossed by a rapid enough wave.
% (Still needed: expression for fixation probability of beneficial mutant--this should be easy enough to obtain.
% Can start with equation 3 of \citet{neher_shraiman_2011} and drop the recombination term, which is what I'm doing now.
% This is tantamount to a ``mean field" approximation but might still work out okay.
% Other possible approach is to use something like the expression of  \citet{desai_fisher_2007} for the fixation probability and rescale it to a traveling wave model.
% Once we have the fixation probability, studying sequential and deterministic fixation will be easier than studying tunneling, since sequential fixation is essentially two different fixation events.)

% One difficulty in computing the rate of tunneling in the high $N\sigma$ case is that the expected bubble size and fixation probability will both turn out to depend on the background fitness of an individual.
% That is to say, the total rate of valley crossing will be given by $\int P(x) \pfix(x,s) \mathbb{E}\left[ w(x,\delta) \right] dx$, with $P(x)$ the (presumably Gaussian) background fitness distribution.
% One cannot simply compute $\pfix$ and the weight distribution by themselves but rather must convolute both of them over the background fitness.
% This is intuitive: individuals near the nose of the distribution will tend to give rise to longer-lived lineages, and because they are near the nose, the complex adaptation will therefore likewise be more likely to fix.
% We proceed to derive both $\pfix(x,s)$ and $P(w)$ in the zero recombination limit.

% The establishment probability $p(x,s)$ obeys the PDE
% \begin{equation}
% v \partial_x p(x,s) = (x+s)p(x,s) - (1+x+s)p(x,s)^2.
% \end{equation}
% To see this, consider the generic expression for the fixation probability
% \begin{equation}
% p(X,t-dt) = p(X,t)-dt(D+B(X,t))p(X,t) + dt(B(X,t)(2p(X,t)-p(X,t)^2)
% \end{equation}
% \citep{barton_1995}. Taking the limit as $dt \to 0$ and inserting $D = 1, B = 1 + X -\bar{X}(t)+s$ yields a PDE for $p(X,t)$.
% Defining $x = X - vt$, we then have $\partial_t p(X-vt) = -v\partial_x p(x)$.

% This is a Bernoulli equation and is readily solved by substituting $u(x,s) = 1/p(x,s)$. We then have that $\partial_x u(x,s) = -\frac{\partial_x p(x,s)}{p(x,s)^2}$, so that
% \begin{equation}
% \partial_x u(x,s) = -\frac{(x+s)}{v}u(x,s) - \frac{1+x+s}{v}.
% \end{equation}
% The use of the integrating factor $e^{\int \frac{x+s}{v} dx} = Ce^{\frac{(x+s)^2}{2v}}$ yields
% \begin{equation}
% Ce^{\frac{(x+s)^2}{2v}} (\partial_x u + \frac{x+s}{v}u) = -Ce^{\frac{(x+s)^2}{2v}}\frac{1+x+s}{v}.
% \end{equation}
% The left side is simply $\partial_x(Ce^{\frac{(x+s)^2}{2v}}u)$, so (after dividing $C$ out) we can integrate both sides, then multiply by $e^{\frac{-(x+s)^2}{2v}}$ to obtain
% \begin{equation}
% u(x,s) =  -e^{\frac{-(x+s)^2}{2v}} \frac{\sqrt{\pi /2} \textrm{erfi}(\frac{x+s}{\sqrt{2v}}}{\sqrt{v}}) - 1 + C(e^{\frac{-(x+s)^2}{2v}}),
% \end{equation}
% with $C$ a constant. Of course, $p = 1/u$.

% Unfortunately it is not clear how to get rid of $C$, as there is no obvious ``initial condition".
% The most we are likely to know about $p$ is that $p$ approaches $0$ in the limit of low $x$, but behaves $\sim s$ in the limit of high $x$ before saturating at $1$ (however, there are unlikely to be any individuals in the population at such high $x$).
% One possibility is to impose an \emph{ersatz} ``solvability condition": surely the fixation probability of a neutral mutation averaged over the entire population must equal $1/N$. We would then have
% \begin{equation}
% \int_{-\infty}^{\infty} P(x)p(x,0)dx = \frac{1}{\sqrt{2\pi}\sigma}\int_{-\infty}^{\infty} \frac{e^{-x^2/2\sigma^2}}{-e^{-x^2/2\sigma^2} \frac{\sqrt{\pi /2} \textrm{erfi}(\frac{x}{\sqrt{2\sigma^2}})}{\sigma} - 1 + Ce^{-x^2/2\sigma^2}}dx = \frac{1}{N}.
% \end{equation}
% Multiplying the top and bottom of the integrand by $e^{x^2/2\sigma^2}$ would clean this up, yielding
% \begin{equation}
% \frac{1}{\sqrt{2\pi}\sigma}\int_{-\infty}^{\infty} \frac{1}{-\frac{\sqrt{\pi /2} \textrm{erfi}(\frac{x}{\sqrt{2\sigma^2}})}{\sigma} - e^{x^2/2\sigma^2} + C}dx = \frac{1}{N}.
% \end{equation}
% Unfortunately this integral has an imaginary error function in the denominator, so there is not likely to be a closed form expression: we could explore it numerically and see where it peaks, however.

% Another idea would be to consider an alternative way of solving the differential equation for $p$.
% Return to
% \begin{equation}
% v \partial_x p(x,s) = (x+s)p(x,s) - (1+x+s)p(x,s)^2.
% \end{equation}
% Suppose we make the \emph{ansatz} that $x+s \ll 1$, and define $\chi = x/\sigma$, $\tilde{s} = s/\sigma$, and $\tilde{p}(\chi,\tilde{s}) = p(x,s)/\sigma$. Then
% \begin{equation}
% \partial_\chi \tilde{p}(\chi,\tilde{s}) = (\chi + \tilde{s})\tilde{p}(\chi,\tilde{s}) - \tilde{p}(\chi,\tilde{s})^2.
% \end{equation}
% Now define $q(\chi,\tilde{s})$ such that $\tilde{w}(\chi,\tilde{s}) = \partial_x \log q(\chi,\tilde{s})$.
% Then
% \begin{equation}
% \partial_\chi^2 q(\chi,\tilde{s}) - (\chi+\tilde{s})\partial_\chi q(\chi, \tilde{s}) = 0.
% \end{equation}
% Defining $\vartheta = \chi + \tilde{s}$ and $q(\vartheta) = e^{\vartheta^2/4}h(\vartheta)$ yields
% \begin{equation}
% \partial_\vartheta^2 h(\vartheta) - (\frac{1}{2} + \frac{\theta^2}{4})h(\vartheta) = 0,
% \end{equation}
% which is the parabolic cylinder equation.
% We then need to find the form of $q$ that has the correct asymptotics.
% This does have a solution:
% \begin{equation}
% q(\chi) = \int_0^\infty d\lambda e^{\vartheta \lambda - \lambda^2/2}/\lambda.
% \end{equation}

% Next, we seek the weight distribution as a function of $x$.
% Let $p(w,T|k,t,x)$ be the probability that a bubble has attained a weight of $w$ at time $T$, given that there were $k$ individuals in the lineage at time $t$.
% Ultimately what we will be interested in is the Laplace transform $\hat{p} (z,T|k,t,x) = \int dz e^{-zw} p(w,T|k,t,x)$.
% Here $w$ represents (as before) the weight of a mutant bubble and evolves according to
% \begin{align*}
% -(\partial_t -k\partial_w) p(w,T|k,t,x) = &-k(2+x-\bar{x}+s)p(w,T|k,t,x) \\
% & + k(1+x-\bar{x}+s)p(w,T|k+1,t,x) \\
% & +kp(w,T|k-1,t,x). \\
% \end{align*}
% We have that $\hat{p} (z,T|k,t,x) = \hat{p}^k (z,T|1,t,x)$, by the assumption that the behavior of each of the $k$ copies of the mutant allele is independent of the other.
% Recall that, in the traveling wave approximation, $\bar{x} = vt = \sigma^2 t$.
% The notation here can be cleaned up by defining $\phi = 1 - \hat{p}$, as well as $\tilde{s} = s/\sigma$, $\chi = x/\sigma - \sigma T$, $\tau = \sigma(T-t)$, and $\theta = \chi+\tau+\tilde{s}+z$.
% Rewriting the preceding in terms of $\phi$ (note that differentiating $p$ with respect to $w$ simply pulls out a factor of $z$) yields
% \begin{equation}
% \partial_\tau \phi(\tau,z,\chi) = z + \theta \phi(\tau,z,\chi) - \phi^2(\tau,z,\chi).
% \end{equation}
% Unfortunately $\theta$ depends on $\tau$, so we will have to be careful here.
% This is a Riccati equation.
% Let $\phi_0(\tau,z,\chi)$ be a particular solution for $\phi$: then the general solution becomes
% \begin{equation}
% \phi(\tau,z,\chi) = \phi_0(\tau,z,\chi) + \frac{\Phi(\tau,z,\chi)}{C + \int \Phi(\tau,z,\chi) d\tau},
% \end{equation}
% where $\Phi(\tau,z,\chi) = e^{\int (-2\phi_0 + \theta ) d\tau}$.
% It remains to find a particular solution of $\phi$.
% We can ``guess" a steady state solution, in which case $0 = \phi^2 - \theta \phi - z = 0$: this corresponds to $\phi_0 = \frac{\theta \pm \sqrt{\theta^2 + 4z}}{2}$.
% Computing the exponent in the integral (and changing integration variables from $\tau$ to $\theta$) yields
% \begin{equation}
% \Phi (\tau, z, \chi) = e^{-\int \sqrt{\theta^2 + 4z} d\theta} = e^{\theta\sqrt{\theta^2+4z} + \frac{4z}{2}\log (\theta + \sqrt{\theta^2 + 4z})}.
% \end{equation}
% (Still need to solve this and carefully consider the integral boundary.)

% We now turn to the case where $\rho > 0$.
% Because we are explicitly dealing with crossovers, the results of \citet{neher_kessinger_2013} are useful here.
% We consider the genome as divided into a series of effectively asexual ``blocks" in which the recombination rate is very low.
% The size of each block is determined by a balance between recombination, which chops up the blocks, and the rate of adaptation, which amplifies them.
% The size of such a block is not a parameter of our model but is given by $\xi_b = \frac{\sigma^2}{2L\rho^2 c \log N\sigma_b}$, with $L$ the number of loci, $c$ a constant of order $1$, and $\sigma_b$ the proportion of the fitness variance segregating within the block.
% If $\xi_b$ is large enough that both focal loci fall within the same ``block", then an analysis similar to that of Neher and Shraiman (2009) may be appropriate: the two loci effectively segregate within the same traveling wave.
% If not, then the value of $N\sigma_b$ becomes the relevant factor.
% $N\sigma_b \gg 1$ implies that each locus effectively operates within its own traveling wave: in order for the full complex adaptation to arise, establish, and fix, both must be at high enough frequency at the same time for either mutation or recombination to produce a double mutant.
% On the other hand, $N\sigma_b \ll 1$ implies that each locus effectively evolves on a neutral genetic background neutrally, and we recover the dynamics of genetic drift.
% \fi

\bibliographystyle{genetics}
\bibliography{bib}

\end{spacing}

\end{document}