\documentclass[10pt]{revtex4}
%\documentclass{article}

\usepackage{amsmath}
\usepackage{amssymb}
\usepackage{graphicx}
\usepackage{mathtools}
\usepackage{subcaption}
\captionsetup{compatibility=false}

\title{Supplemental information for ``Genetic draft and valley crossing"}
%\author[$\ast$]{Taylor Kessinger}
%\author[$\ast$,1]{Jeremy Van Cleve}
%\affil[$\ast$]{Department of Biology, University of Kentucky}


\begin{document}

\maketitle

\begin{figure}[h]
\begin{subfigure}[d]{0.48\textwidth}
\includegraphics[width=\textwidth]{Figures/{var_UL_5.0e-8}.pdf}
\end{subfigure}
\begin{subfigure}[d]{0.48\textwidth}
\includegraphics[width=\textwidth]{Figures/{var_UL_0.01}.pdf}
\end{subfigure}
\begin{subfigure}[d]{0.48\textwidth}
\includegraphics[width=\textwidth]{Figures/{var_UL_0.05}.pdf}
\end{subfigure}
\begin{subfigure}[d]{0.48\textwidth}
\includegraphics[width=\textwidth]{Figures/{var_UL_0.1}.pdf}
\end{subfigure}
\caption{Recreation of figure 5b from \citet{weissman_2009} with varying $\sigma$ and $UL$. Varying $UL$ by several orders of magnitude has no obvious, systematic effect on the crossing time.}

\end{figure}

\begin{figure}[h]
\begin{subfigure}[d]{0.48\textwidth}
\includegraphics[width=\textwidth]{Figures/{equilibration_0.05_1.0}.pdf}
\end{subfigure}
\begin{subfigure}[d]{0.48\textwidth}
\includegraphics[width=\textwidth]{Figures/{equilibration_0.0005_1.0}.pdf}
\end{subfigure}
\begin{subfigure}[d]{0.48\textwidth}
\includegraphics[width=\textwidth]{Figures/{equilibration_5.0e-6_1.0}.pdf}
\end{subfigure}
\caption{The number of fitness classes over time; the dotted line represents the chosen ``equilibration" cutoff. Additional classes are added past the cutoff only at very low $\sigma$ where, by design, the number of fitness classes is likely irrelevant anyway.}

\end{figure}

\bibliography{bib}

\end{document}