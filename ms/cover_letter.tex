\documentclass[kentucky,10pt,foldmarks=off,backaddress=false,refline=dateleft,letterpaper]{scrlttr2}

\usepackage{xspace}
\usepackage[parfill]{parskip}

% Sender's information
\setkomavar{signature}{Jeremy Van Cleve\\Taylor Kessinger}
%\setkomavar{signature}{\includegraphics[scale=0.3]{hand_sig.pdf}\Jeremy Van Cleve} % use PDF image of signature
\setkomavar{date}{\today}
\setkomavar{subject}{}

\begin{document}

% Recipient's information
\begin{letter}{
    GENETICS Editorial Board\\
    6120 Executive Boulevard\\
    Suite 550\\
    Rockville, MD 20852}

\opening{Dear GENETICS Editorial Board}


We are excited to submit to GENETICS our manuscript entitled ``Genetic draft and valley crossing''.
Our work focuses on the intersection of two topics:
(i) how populations evolve when the mapping between genotype and fitness, the so-called ``fitness landscape'', is rugged with valleys between multiple peaks, and
(ii) how rapid adaptation and the genetic draft that it creates profoundly alter evolutionary processes.
Both topics have received a great deal of interest from geneticists and evolutionary biologists in the last few years with a number of important studies in GENETICS, but no study has looked at how genetic draft affects evolution on rugged fitness landscapes.

Specifically, we look at how the time to cross a fitness valley is affected by the variation in fitness in a population that is adapting to a novel environment.
When that variation is low, genetic drift is the primary stochastic force shaping evolutionary dynamics, and classic valley crossing results, such as Weissman et al. (2009, TPB) and Weissman et al. (2010, GENETICS), hold.
In contrast, genetic draft is more important when fitness variation is high, a condition that obtains in many populations of interest, e.g., pathogens and microbial evolution experiments.
We show that valley crossing is fundamentally different in this case.

We show that genetic draft generally increases the time to cross a fitness valley and hence the time for a complex adaptation to evolve on a rugged fitness landscape.
This is due to the lower efficacy of selection on the complex adaptation in the presence of genetic draft.
Crucially, we also find that complex adaptations are more likely relative to simple selective sweeps under genetic draft and rapid adaptation than under genetic drift.
This means that populations evolving in novel environments or with pathogens may more easily construct complex adaptations than populations in constant environments.
This result suggests that genetic draft may be an important and underappreciated mechanism in the evolution of complex traits.

We suggest the following reviewers for our manuscript:\\
Daniel Weissman (Emory University), Joshua Plotkin (University of Pennsylvania), Andreas Wagner (University of Zürich), and Ben Good (UC Berkeley).

We hope you find our work appropriate for GENETICS.
Please let us know if you have any questions.

\closing{Best regards,}

\end{letter}

\end{document}